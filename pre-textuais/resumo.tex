\begin{resumo}
    %remova estas linhas e troque pelo seu conteúdo
    O resumo na língua vernácula é um elemento pré-textual obrigatório. O resumo deve ser elaborado conforme a ABNT NBR 6028.O resumo de monografia deve ter entre 150 e 500 palavras em um único parágrafo. O resumo deve conter a contextualização do problema, o “gap” ou lacuna no estado da técnica, a solução proposta ou objetivos do trabalho, a abordagem ou metodologia empregada, os principais resultados, breves discussão e conclusão. Vejam as videoaulas do Professor Zucolotto \cite{Zucolotto} e as notas de aula da disciplina de Metodologia da Pesquisa para exemplos e detalhes de como fazer um bom resumo.
    
    %troque por suas palavras-chave
    \palavraschave{palavra-chave 1, palavra-chave 2, palavra-chave 3}
    %[entre 3 e seis palavras-chave, em ordem alfabética, separadas por ponto e vírgula e finalizadas por ponto; esta formatação diverge da apresentada na norma ABNT NBR 6028]
\end{resumo}