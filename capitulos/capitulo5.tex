\chapter{DISCUSSÃO}
\thispagestyle{empty}
Este capítulo é o âmago da monografia. Todos os capítulos são importantes, claro, e devem ter seu conteúdo rigorosamente trabalhado. Entretanto, é neste capítulo que o autor demonstra todo conhecimento adquirido sobre o assunto e o domínio que ele tem sobre os resultados obtidos. Uma boa discussão deixa o leitor ciente que o autor domina o tema e é de fato “proprietário” do conhecimento desenvolvido. Aqui, o autor deve mostrar que as hipóteses foram verificadas e que os objetivos propostos foram atingidos evidenciando sua contribuição ao conhecimento. Esta é a parte da monografia na qual o autor coloca sua opinião sobre o tema e discute com seus pares, por meio do que existe de mais atual na literatura.

Uma boa e completa discussão mescla um debate sobre os seus resultados, realçando pontos fortes e fracos, com uma comparação fundamentada com o que há na literatura. É importante mencionar que não é obrigatório que os resultados da pesquisa sejam idênticos aos reportados na literatura especializada. O autor deve ter senso crítico para interpretar e discutir o que há de semelhante, assim como tranquilamente apontar as diferenças obtidas em relação a outros autores.

A discussão não é uma repetição da revisão bibliográfica, mas é normal usar algumas referências lá citadas para dialogar com os resultados da pesquisa. Uma boa estratégia é discutir os dados obtidos à luz das referências: os dados confirmam a teoria envolvida? Eles sugerem outras interpretações? São sugeridas particularidades?

Há uma questão algoz na redação de monografias de cunho técnico: os capítulos resultados e discussão estarem juntos. É possível, claro. O modelo aqui apresentado se enquadra melhor às pesquisas para as quais há bastante literatura a respeito, o que nem sempre é o caso. Caso o tema tenha literatura importante publicada, a discussão não é sobre os resultados, mas trata-se da comparação dos resultados com a literatura, ou sobre a validação das hipóteses.
