\chapter{Introdução}
%\thispagestyle{empty}

Este modelo deve ser utilizado para facilitar a elaboração da monografia, particularmente teses e dissertações. Este modelo poderá ser utilizado, com as devidas adaptações, para trabalhos de disciplinas, quando exigido.
Neste documento estão sendo utilizadas as definições conforme a norma técnica ABNT NBR 14724:2011. A definição de tese é:
\begin{citacao}
“documento que apresenta o resultado de um trabalho experimental ou exposição de um estudo científico de tema único e bem delimitado. Deve ser elaborado com base em investigação original, constituindo-se em real contribuição para a especialidade em questão. É feito sob a coordenação de um orientador (doutor) e visa a obtenção do título de doutor, ou similar” (ABNT NBR 14724:2011, item 3.33).
\end{citacao}
A definição de dissertação é a seguinte:
\begin{citacao}
“documento que apresenta o resultado de um trabalho experimental ou exposição de um estudo científico retrospectivo, de tema único e bem delimitado em sua extensão, com o objetivo de reunir, analisar e interpretar informações. Deve evidenciar o conhecimento de literatura existente sobre o assunto e a capacidade de sistematização do candidato. É feito sob a coordenação de um orientador (doutor), visando a obtenção do título de mestre” (ABNT NBR 14724:2011, item 3.10).
\end{citacao}
Trabalho de conclusão de curso de graduação, trabalho de graduação interdisciplinar, trabalho de conclusão de curso de especialização ou aperfeiçoamento tem a seguinte definição.
\begin{citacao}
“documento que apresenta o resultado de estudo, devendo expressar conhecimento do assunto escolhido, que deve ser obrigatoriamente emanado da disciplina, módulo, estudo independente, curso, programa, e outros ministrados. Deve ser feito sob a coordenação de um orientador” (ABNT NBR 14724:2011, item 3.35).
\end{citacao}

A definição de monografia é apresentada na norma técnica ABNT NBR 6023:2018, segundo a qual monografia é um “item não seriado, isto é, item completo, constituído de uma só parte, ou que se pretende completar em um número preestabelecido de partes separadas”. Assim sendo, teses, dissertações e TCC são monografias.

Além de ser um modelo para elaboração da monografia, este documento traz dicas e boas práticas de elaboração de textos técnicos. Parte do conteúdo é de utilização obrigatória, como as regras de ortografia e gramática, e parte é recomendação. Cabe ao autor, em entendimento com o orientador, discernir sobre o que é obrigatório e o que é recomendável.

AVISO IMPORTANTE: este guia somente poderá ser utilizado como modelo se a edição for feita na versão atual e oficial do Word, preferencialmente o Office 365. O uso de versões anteriores poderá desconfigurar o texto. O autor deverá ficar atento aos requisitos de formatação, particularmente listas, sumário e referências cruzadas.

O primeiro capítulo da monografia deve conter os seguintes elementos: motivação para a pesquisa; tema; justificativa; premissas; hipóteses, escopo; e objetivos. Não é necessário que esses elementos sejam separados em seções específicas, exceto os objetivos. Um bom texto introdutório deixa clara a motivação que levou o pesquisador a escolher determinado tema, justificando apropriadamente sua realização. As premissas são afirmações, geralmente embasadas na literatura, nas quais as hipóteses são construídas. Por sua vez, hipóteses\footnote{Pesquisas com hipóteses englobam duas ou mais variáveis que se relacionam apenas de uma entre duas formas: associação entre as variáveis ou interferência entre as variáveis \cite{volpato_dicas_2010}. }  são respostas provisórias a uma ou mais perguntas de pesquisa, que ainda não foram testadas. Vale mencionar que premissas não podem ser falhas, ou seja, falsas. Se forem, toda a pesquisa será invalidada. As hipóteses, por sua vez, podem ou não ser comprovadas ao final da pesquisa. Caso alguma hipótese seja refutada no decorrer da pesquisa fundamentada nos resultados experimentais, a pesquisa continuará válida e a discussão e conclusões deverão retratar esse achado.

O escopo é fundamental constar na introdução. Este elemento delimita o campo de atuação ou abrangência da pesquisa. As premissas, inclusive, devem fazer alusão ou serem fundamentadas no escopo definido para a pesquisa. Por exemplo: o estudo da propagação ultrassônica em um meio (líquido ou gasoso, por exemplo) depende fundamentalmente da amplitude da onda ultrassônica. A partir de uma determinada amplitude e em função das características do meio e da frequência do sinal, pode-se adotar a teoria de propagação linear ou não linear. Portanto, as premissas, incluindo as equações que vão ser utilizadas, dependem do escopo, ou seja, propagação linear ou não linear. Entretanto, se for definido o escopo como propagação linear, mas se forem realizados experimentos com grandes amplitudes, as premissas serão falsas porque o escopo da pesquisa não foi respeitado.

Em alguns casos, a introdução poderá conter parte da revisão bibliográfica, embora este guia proponha um capítulo independente para revisão bibliográfica (ou fundamentação teórica). 
Cabe ao orientador e o discente escolher a melhor forma de introduzir e fundamentar sua monografia. Segundo a norma ABNT NBR 14724, os elementos textuais são divididos apenas como “Introdução”, “Desenvolvimento” e “Conclusão”, ficando a critério do autor a nomenclatura desses elementos. Reforçando, este guia propõe uma estrutura do “Desenvolvimento” composta por “Fundamentação teórica” (ou “Revisão bibliográfica”), “Metodologia” (ou “Materiais e métodos”), “Resultados” e “Discussão”.

\section{MODELO DO WORD}

Para facilitar o emprego deste modelo, foram criados estilos de texto para serem aplicados nas monografias e para facilitarem a uniformização dos textos. A seguir são apresentados alguns dos estilos criados para este modelo:
TEXTO NORMAL: Fonte: (Padrão) Times New Roman, 12 pt, Recuo: Primeira linha:  1 cm, Justificado. Espaçamento entre linhas:  1,5 linhas, Espaço Antes:  6 pt.
CITAÇÃO LITERAL: Fonte: (Padrão) Times New Roman, 11 pt, Recuo: À esquerda: 4 cm, Justificado. Espaçamento entre linhas:  1,5 linhas, Espaço Depois de:  0 pt,
TÍTULO 1: Fonte: Times New Roman, 12 pt, Negrito, Recuo: À esquerda:  0 cm; Deslocamento:  1 cm, Vários níveis + Nível: 1 + Estilo da numeração: 1, 2, 3, … + Iniciar em: 1 + Alinhamento: Esquerda + Alinhado em:  0 cm + Recuar em:  1 cm, Prioridade: 100.
Outros estilos utilizados neste modelo são: FIGURA; FONTE ORG; ILUSTRAÇÃO; QUADRO; REF BIBLIOGRÁFICAS; TABELA; TÍTULO NN; TÍTULO 2; TÍTULO 3.


\section{Paginação}

Todas as folhas, a partir da folha de rosto inclusive, devem ser contadas sequencialmente, mas não numeradas. A Capa não faz  parte desta numeração. Use a funcionalidade “quebra de seção” para separar a capa do restante do texto. Este guia, se usado como modelo, já está com esta formatação.

A numeração é inserida a partir da primeira página da parte textual (“Introdução”), em algarismos arábicos, no canto superior direito da página, a 2 cm da borda superior, ficando o último algarismo a 2 cm da borda direita da página. No caso de o trabalho ser constituído de mais de um volume, deve ser mantida uma única sequência de numeração das folhas, do primeiro ao último volume. Havendo apêndice e anexo, as suas páginas devem ser numeradas de maneira contínua e a paginação deve dar seguimento à do texto principal.

Quando o trabalho for digitado em anverso e verso, a numeração das páginas deve ser colocada sempre no canto superior externo, ou seja, no anverso no canto superior direito e no verso no canto superior esquerdo.


\section{Objetivo}

Toda pesquisa deve ter um objetivo principal ou geral. O objetivo geral pode ser detalhado em objetivos específicos. Observe que os objetivos específicos fazem parte do objetivo principal. Não são, portanto, outros objetivos distintos. Não utilize o termo “objetivos secundários” pois transmite a impressão de serem pouco relevantes, o que não é o caso. O objetivo geral caracteriza de forma resumida a finalidade do projeto, descrito em um único parágrafo. Ele deve expressar de forma clara qual é a intenção (finalidade) daquele projeto de pesquisa que descreve e delimitar qual será o escopo do trabalho.

Uma pesquisa com muitos objetivos não é uma pesquisa objetiva! Portanto, selecione entre 3 e 6 objetivos específicos. Vale mencionar que cada objetivo específico deve contemplar uma ou algumas etapas do desenvolvimento do projeto. Os objetivos específicos para serem alcançados precisam ter um método ou metodologia, recursos para serem alcançados, devem gerar um resultado (entrega) e devem ser discutidos apropriadamente ao final do trabalho.

Os objetivos devem ser iniciados por verbo. Veja a seguir alguns exemplos categorizados de verbos que podem ser empregados para iniciarem os objetivos (geral ou específicos).


\begin{itemize}
    \item \underline{Verbos de conhecimento:} associar; calcular; citar; classificar; definir; descrever; distinguir; enumerar; especificar; enunciar; estabelecer; exemplificar; expressar; identificar; indicar; medir; mostrar; nomear; registrar; relacionar; relatar; selecionar.
    \item \underline{Verbos de compreensão:} concluir; descrever; distinguir; deduzir; demonstrar; discutir; explicar; identificar; ilustrar; inferir; interpretar; localizar; relatar; revisar.
    \item \underline{Verbos de aplicação:} aplicar; classificar; estruturar; ilustrar; interpretar; organizar; relacionar.
    \item Verbos de análise: analisar; classificar; categorizar; combinar; comparar; comprovar; constatar; correlacionar; diferenciar; discutir; detectar; descobrir; descriminar; examinar; identificar; investigar; provar; selecionar.
    \item \underline{Verbos de síntese:} sintetizar; combinar; compor; criar; comprovar; deduzir; desenvolver; documentar; explicar; organizar; planejar; relacionar
    \item \underline{Verbos de avaliação:} avaliar; concluir; constatar; criticar; interpretar; julgar; justificar; padronizar; relacionar; selecionar; validar; valorizar.
\end{itemize}
