\chapter{METODOLOGIA [MATERIAIS E MÉTODOS]}
%\thispagestyle{empty}	
\section{CONTEÚDO DESTE CAPÍTULO}
Neste capítulo devem ser apresentados todos os recursos utilizado na sua pesquisa. Se o desenvolvimento do método fez parte da sua pesquisa, então o capítulo pode ser denominado “metodologia”, que significa “estudo dos métodos”. Quando um método foi desenvolvido ou aprimorado, então ele passou por um processo de validação de método, portanto deve ser apresentado neste capítulo.

Há trabalhos que, de fato, fazem um estudo dos métodos, para escolher ou propor algum. Quando isto é feito, precisa estar claro que se trata de metodologia. Este estudo pode ser feito sem o uso de materiais ou aplicação de algum método. Por outro lado (mais comum), não há estudo de métodos. Simplesmente se adota algum método, cuja aplicação prescinde de algum material. Por fim, pode haver a situação onde se discute a metodologia, se escolhe algum método e se aplica através dos materiais necessários.

\section{DESCREVA O QUE USOU E USE O QUE DESCREVEU}
Alguns erros recorrentes presentes em muitas monografias estão na inadequação da apresentação dos materiais e métodos. É importante para uma ótima transmissão de conhecimento descrever TUDO que foi utilizado na sua pesquisa neste capítulo, incluindo aplicativos, recursos humanos, insumos, instrumentos de medição, padrões, equipamentos e acessórios. É considerado erro grave utilizar algum recurso e apresentá-lo em outros capítulos, por exemplo no referencial teórico ou nos resultados. Se policie para não fazer isso.

Muitas vezes o pesquisador propõe um certo método e, no decorrer da pesquisa, termina optando por outra abordagem. Outro fato recorrente é prever o emprego de um determinado instrumento de medição mas utilizar outro semelhante. Caso algo semelhante aconteça, o correto é apresentar ambos instrumentos neste capítulo, eventualmente justificando a troca.

\section{INSTRUMENTOS DE MEDIÇÃO E PADRÕES}

Para apresentar um equipamento ou instrumento de medição, ou mesmo um insumo ou consumível, o correto é a seguinte padronização: “Para realizar a medição XXX foi utilizado um [nome do instrumento, equipamento, padrão ou insumo] modelo [nome, tipo ou descrição do modelo] ([Nome do fabricante], [País]) com as seguintes características: [descrever as principais características]”. Os itens em amarelo devem ser substituídos para cada instrumento e medição, equipamento, padrão ou insumo utilizado na pesquisa.

\section{TRATAMENTO ESTATÍSTICO}

Toda pesquisa de cunho tecnológico demanda algum tratamento estatístico. É obrigatório descrever neste capítulo qual ou quais métodos estatísticos foram empregados para tratamento dos dados ou análise dos resultados. Inclua a probabilidade de abrangências, critérios de aceitação, validação de métodos de medição, testes de hipóteses, técnicas de amostragem, erros máximos admissíveis, tolerâncias e outras limitações estatísticas impostas ou decorrentes do objeto estudado.

\section{USO DE QUESTIONÁRIOS}
Tem sido cada vez mais comum o uso de questionários como instrumentos de pesquisa qualitativa. O uso de questionários requer uma série de cuidados prévios, tanto na elaboração e aplicação do questionário, quanto na análise dos resultados obtidos. Segundo \citeonline{Melo2015}, “a utilização indevida de um questionário, ou um questionário mal formulado, pode resultar na geração de informações equivocadas e causar erros de conclusões, afetando a validade do estudo”.

Os questionários devem ser elaborados de forma criteriosa, estabelecendo uma ligação com o problema de pesquisa, as premissas (ou hipóteses) do estudo, a população a ser pesquisada e os métodos de análises de dados disponíveis. As perguntas que comporão o questionário devem ser claras e objetivas, minimizando ao máximo a possibilidade de vieses por aqueles que irão respondê-las, ou seja, isentas de ambiguidades. Desse modo, perguntas que induzem a resposta (p.ex. “você escova os dentes todos os dias?”), que não trazem a informação pretendida (perguntas que usam pronomes indefinidos: algum, nenhum, todo, outro, muito, pouco, certo, vários, tanto, quanto, qualquer, alguém, ninguém, tudo, outrem, nada, quem, cada, algo) e perguntas que se auto respondem (p.ex. “você prefere que o expediente termine mais cedo?”) não devem ser utilizadas. Deve-se ter atenção também com a lógica na apresentação sequencial das perguntas. \citeonline{Melo2015} acrescentam que perguntas que sugiram ou condicionem a resposta, que possuam conteúdo emocional, que levem o respondente à necessidade de fazer cálculos, que façam alusão a nomes que impliquem em aceitação ou rejeição e que contagiem outras respostas, devem ser evitadas.

É importante frisar que um questionário possui um tamanho ótimo, não devendo ser demasiadamente longo ou curto. Questionários muito longos geram enfado nos respondentes, que ou não o responderão ou passarão a fazê-lo de forma não criteriosa, de qualquer maneira. Questionários muito curtos podem ser insuficientes em termos dos dados necessários para a pesquisa. Nesse ponto, pode-se utilizar a estatística multivariada (nomeadamente a análise multifatorial) a fim de determinar quais são as perguntas importantes a serem mantidas no questionário e quais podem ser removidas.
Ao se optar pelo uso de questionários, o pesquisador deve levar em consideração o tamanho da população e consequentemente da amostra a qual será aplicado este instrumento. Se a população for muito grande, o erro amostral e o tempo na execução da pesquisa devem ser levados em conta antes do seu início. De forma geral, a taxa de retorno dos questionários respondidos no período da pesquisa é baixa.

Outro ponto importante a ser considerado é a confiabilidade do questionário como instrumento de pesquisa. Imagine que você está fazendo uma pesquisa que envolve a medição de pH: você confiaria nos dados obtidos por meio de um medidor de pH que não estivesse devidamente calibrado? Assim também é acontece com os questionários. É necessário que os questionários tenham sua confiabilidade avaliada, o que pode ser feito de diversas maneiras, como por meio do cálculo do alfa de Cronbach, teste-reteste (administrar o questionário duas vezes na mesma pessoa com dado intervalo de tempo), confiabilidade Inter Avaliador (Concordância) (avaliada, por exemplo, por meio do cálculo do Tau de Kendall, correlação interclasse etc.). Alguns fatores podem influenciar a confiabilidade dos questionários. \citeonline{Freitas2005} mencionam três fatores: o número de itens, o tempo de aplicação do questionário e a amostra de avaliadores. 
Por fim, mas não menos importante, ao se elaborar um questionário deve-se avaliar quais os tipos de questões (abertas fechadas ou mistas) bem como qual a escala de avaliação (ordinal, nominal, intervalar etc.) a serem utilizadas. Um tipo de escala muito popular é a escala Likert. A escala Likert típica é uma escala ordinal de 5 ou 7 pontos usada pelos entrevistados para avaliar o grau em que eles concordam ou discordam de uma afirmação.
