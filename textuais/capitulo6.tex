\chapter{CONCLUSÃO}
%\thispagestyle{empty}

Toda monografia deve terminar com uma conclusão. Neste capítulo não se deve simplesmente repetir trechos da discussão, mas deve ser evidenciada a contribuição da pesquisa para a evolução do conhecimento sobre o tema estudado à luz dos resultados experimentais obtidos.

Embora não seja obrigatório, em geral as conclusões de monografias terminam com um parágrafo elencado assuntos, temas ou abordagens que poderiam servir como trabalhos futuros, quer seja pelo próprio autor ou por outros pesquisadores.
\section{PROPOSTA DE CONTINUIDADE [OPCIONAL]}

Caso haja atividades relevantes relacionados ao trabalho desenvolvido, eles são listados com uma breve explicação, neste capítulo. Afinal, a pesquisa científica é incremental e não se pode esgotar o assunto em um único trabalho de pesquisa.
